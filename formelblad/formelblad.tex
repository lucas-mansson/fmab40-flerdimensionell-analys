\author{Lucas Månsson}
\documentclass[a4paper,12pt]{article}
\usepackage[letterpaper,top=2cm,bottom=2cm,left=3cm,right=3cm,marginparwidth=1.75cm]{geometry}
\usepackage[utf8]{inputenc}
\usepackage{amsmath, amssymb}
\usepackage{float}
\usepackage{parskip}
\usepackage{graphicx} 

\graphicspath{ {images/} }

\title{Flerdimensionell analys \\ Formelblad och anteckningar LP1 2025}
\date{}

\begin{document}
\maketitle

\section{Kapitel 1: Grundläggande begrepp}

\subsection{Mängder och tallinjen $\mathbb{R}$}
Snitt, union och differens:
\begin{figure}[H]
  \centering
  \includegraphics[width=1\textwidth]{snittuniondifferens.png}
  \caption{}
\end{figure}

\[
A \cup B, \quad A \cap B, \quad A \setminus B
\]

Absolutbelopp:

\[
    |ab| = |a||b|, \quad |\frac{a}{b}| = \frac{|a|}{|b|} \quad |a+b| \leq |a| + |b|
\]

\subsection{Planet $\mathbb{R}^2$ och rummet $\mathbb{R}^3$}
Avståndsformel.  
\[
    d = \sqrt{(x_1 - x_2)^2 + (y_1 - y_2)^2}
\]

\subsubsection*{Mängder i planet och rummet}
\textbf{Omgivning:}  
Med en omgivning av punkten $(a, b)$ i planet menar vi alla punkter i en cirkelskiva kring denna. Detta kan uttryckas:

\[
|(x,y) - (a,b)| < d
\]

Notera att den stränga olikheten ovan innebär att punkterna på själva cirkeln inte ingår i omgivningen.

\textbf{Öppen och sluten mängd.}

\subsection{Begrepp och metoder från linjär algebra}

\subsubsection*{Vektorer}
\[
u + v = (a_1,b_1) + (a_2,b_2) = (a_1+a_2, \, b_1+b_2)
\]
\[
u - v = (a_1,b_1) - (a_2,b_2) = (a_1-a_2, \, b_1-b_2)
\]
\[
\lambda u = \lambda (a_1, b_1) = (\lambda a_1, \lambda b_1)
\]
\[
|u+v| \leq |u| + |v|
\]

\subsubsection*{Skalärprodukt och vektorprodukt}
Skalärprodukt:
\[
u \cdot v = |u||v|\cos\theta
\]
\[
u \cdot v = (a_1, b_1)\cdot(a_2, b_2) = a_1a_2 + b_1b_2
\]

\subsubsection*{Ortogonal projektion}
En ortogonal projektion kan beräknas med projektionsformeln.
\begin{figure}[H]
  \centering
  \includegraphics[width=1\textwidth]{ortogonalprojektion.png}
  \caption{}
\end{figure}
\[
    \mathbf{u'} =  \frac{\mathbf{u} \cdot \mathbf{v}}{\|\mathbf{v}\|^2} \mathbf{v}
\]

\subsubsection*{Linjer}
Vi kan beskriva en linje i planet om vi känner till en punkt $P$ på linjen och en riktningsvektor $v$ som anger dess riktning.  

Om $P=(x_0, y_0)$ och $v = (v_1, v_2)$ så blir linjens ekvation i parameterform:
\[
(x, y) = (x_0, y_0) + t(v_1, v_2)
\]

\textbf{Normalvektor:}  
Varje linje i planet kan beskrivas på normalform:
\[
ax + by + c = 0
\]

Om vi plockar ut koefficienterna framför $x$ och $y$ och bildar vektorn $n = (a,b)$ blir $n$ vinkelrät mot linjen.  

Givet en punkt $P=(x_0,y_0)$ och normalvektor $n=(a,b)$:
\[
a(x - x_0) + b(y - y_0) = 0
\]

\subsubsection*{Plan}
Med hjälp av en punkt och två icke-parallella riktningsvektorer kan man få ett plan på parameterform i rummet.

\subsubsection*{Matriser och determinanter}

\subsection{Rummet $\mathbb{R}^n$}


\section{Kapitel 2: Analytisk geometri}

\subsection{Geometri i $\mathbb{R}^2$}
\subsubsection*{Sammanfattning av första- och andragradskurvor:}
\textbf{Rät linje:}

\begin{figure}[H]
  \centering
  \includegraphics[width=0.25\textwidth]{ratlinje.png}
  \caption{}
\end{figure}
\[
ax + by + c = 0
\]

Parabel:
\begin{figure}[H]
  \centering
  \includegraphics[width=0.25\textwidth]{parabel.png}
  \caption{}
\end{figure}
\[
y = ax^2
\]

Cirkel:
\begin{figure}[H]
  \centering
  \includegraphics[width=0.25\textwidth]{cirkel.png}
  \caption{}
\end{figure}
\[
x^2 + y^2 = r^2
\]

Ellips
\begin{figure}[H]
  \centering
  \includegraphics[width=0.25\textwidth]{ellips.png}
  \caption{}
\end{figure}
\[
\frac{x^2}{a^2} + \frac{y^2}{b^2} = 1, \quad \text{asymptoter: } y = \pm \frac{b}{a}x
\]

Hyperbel
\begin{figure}[H]
  \centering
  \includegraphics[width=0.25\textwidth]{hyperbel.png}
  \caption{}
\end{figure}
\[
\frac{x^2}{a^2} - \frac{y^2}{b^2} = 1, \quad \text{asymptoter: } y = \pm \frac{b}{a}x
\]
För en hyperbel med höger-vänster öppen:
\[
\frac{y^2}{b^2} - \frac{x^2}{a^2} = 1
\]

\subsection{Geometri i $\mathbb{R}^3$}
\subsubsection*{Sammanfattning av första- och andragradskurvor}

\textbf{Plan}
\[
ax + by + cz + d = 0
\]
\begin{figure}[H]
  \centering
  \includegraphics[width=0.25\textwidth]{plan.png}
  \caption{}
\end{figure}

\textbf{Paraboloid}
\[
z = \frac{x^2}{a^2} + \frac{y^2}{b^2}
\]
\begin{figure}[H]
  \centering
  \includegraphics[width=0.25\textwidth]{paraboloid.png}
  \caption{}
\end{figure}

\textbf{Kon}
\[
z^2 = \frac{x^2}{a^2} + \frac{y^2}{b^2}
\]
\begin{figure}[H]
  \centering
  \includegraphics[width=0.25\textwidth]{kon.png}
  \caption{}
\end{figure}

\textbf{Sfär}
\[
x^2 + y^2 + z^2 = r^2
\]
\begin{figure}[H]
  \centering
  \includegraphics[width=0.25\textwidth]{sfar.png}
  \caption{}
\end{figure}

\textbf{Ellipsoid}
\[
\frac{x^2}{a^2} + \frac{y^2}{b^2} + \frac{z^2}{c^2} = 1
\]
\begin{figure}[H]
  \centering
  \includegraphics[width=0.25\textwidth]{ellipsoid.png}
  \caption{}
\end{figure}

\textbf{Hyperbolisk paraboloid}
\[
z = \frac{x^2}{a^2} - \frac{y^2}{b^2}
\]
\begin{figure}[H]
  \centering
  \includegraphics[width=0.25\textwidth]{hyperboliskparaboloid.png}
  \caption{}
\end{figure}

\textbf{Hyperboloid (enmantlad)}
\[
\frac{x^2}{a^2} + \frac{y^2}{b^2} - \frac{z^2}{c^2} = 1
\]
\begin{figure}[H]
  \centering
  \includegraphics[width=0.25\textwidth]{enmantladhyperboloid.png}
  \caption{}
\end{figure}

\textbf{Hyperboloid (tvåmantlad)}
\[
-\frac{x^2}{a^2} - \frac{y^2}{b^2} + \frac{z^2}{c^2} = 1
\]
\begin{figure}[H]
  \centering
  \includegraphics[width=0.25\textwidth]{tvamantladhyperboloid.png}
  \caption{}
\end{figure}

\subsection{Polära och rympolära koordinater}
\subsubsection*{Polära koordinater}
En punkt $P$ i planet med rätvinkliga koordinater $(x, y)$, 
kan också beskrivas med avståndet r från origo tillsammans med vinkel $\varphi$ mot positiva x-axeln.
$P$ har då polära koordinaterna $(r, \varphi)$. 
\begin{figure}[H]
  \centering
  \includegraphics[width=0.25\textwidth]{polarakoordinater.png}
  \caption{}
\end{figure}
\[
\begin{cases}
x = r \sin \varphi, \\
y = r \sin \varphi 
\end{cases}
\]
I de fall vi har annan medelpunkt än origo, ex. $(x_0, y_0)$:
\[
\begin{cases}
x = x_0 + r \sin \varphi, \\
y = y_0 + r \sin \varphi 
\end{cases}
\]
Om vi vill beskriva en ellipsskiva, snarare än bara en ellips:
\[
\begin{cases}
x = x_0 + a r \sin \varphi, \\
y = y_0 + a r \sin \varphi 
\end{cases}
\]

\subsubsection*{Cylindriska och rymdpolära koordinater}
Cylindriska koordinater:
\begin{figure}[H]
  \centering
  \includegraphics[width=0.5\textwidth]{cylindriskakoordinater.png}
  \caption{}
\end{figure}
\[
\begin{cases}
x = r \sin \varphi, \\
y = r \sin \varphi, \\
z = z
\end{cases}
\]

En annan koordinat än $z$ kan vara oförändrad.

Rymdpolära koordinater:
\begin{figure}[H]
  \centering
  \includegraphics[width=0.5\textwidth]{rymdpolarakoordinater.png}
  \caption{}
\end{figure}
\[
\begin{cases}
x = r \sin \theta \cos \varphi, \\
y = r \sin \theta \sin \varphi, \\
z = r \cos \theta
\end{cases}
\]

\section{Kapitel 3: Funktioner}

\subsection{Reellvärda funktioner}
En reellvärd funktion är av typen
\[
\mathbb{R}^n \to \mathbb{R}
\]
dvs. en funktion av $n$ variabler där varje funktionsvärde är reellt.

\subsubsection*{Funktioner av typen $\mathbb{R}^2 \to \mathbb{R}$}
En funktion $f$ av två variabler består av en definitionsmängd $D_f \subseteq \mathbb{R}^2$ och en avbildningsregel:
\[
(x,y) \in D_f \mapsto f(x,y) \in \mathbb{R}
\]

\subsubsection*{Nivåkurvor och nivåytor}
En nivåkurva består av samtliga punkter i $xy$-planet som ger samma funktionsvärde. 
Låt f vara en funktion av två variabler, och C, en konstant. Mängden i xy - planet som ges av ekvationen:
\[
    f(x,y) = C
\]
kallas en \textbf{nivåkurva} till f. Konstanten C motsvarar således “höjden över xy-planet”.
\begin{figure}[H]
  \centering
  \includegraphics[width=0.25\textwidth]{nivakurva.png}
  \caption{}
\end{figure}

\subsection{Vektorvärda funktioner}
En funktion av typen $f: \mathbb{R}^n \to \mathbb{R}^p$, där $p \ge 2$, kallas \textbf{vektorvärd}.
eftersom funktionsvärdena då är vektorer.
\subsubsection*{Funktioner av typen $\mathbb{R} \to \mathbb{R}^2$ och $\mathbb{R} \to \mathbb{R}^3$ (kurvor)}
\[
    \textbf{r}(t) = (x(t), y(t))
\]
\[
    \textbf{r}(t) = (x(t), y(t), z(t))
\]

\subsubsection*{Funktioner av typen $\mathbb{R}^2 \to \mathbb{R}^3$ (ytor)}
\[
    \textbf{r}(s, t) = (x(s, t), y(s, t), z(s, t))
\]

\subsubsection*{Funktioner av typen $\mathbb{R}^2 \to \mathbb{R}^2$ och $\mathbb{R}^3 \to \mathbb{R}^3$ (koordinatbyten)}
(fyll i senare)

\subsubsection*{Funktioner av typen $\mathbb{R}^2 \to \mathbb{R}^2$ och $\mathbb{R}^3 \to \mathbb{R}^3$ (vektorfält)}
(fyll i senare)

\subsection{Sammansättning av funktioner}
\[
    (f \circ g)(x) = f(g(x))
\]

\subsection{Gränsvärden och kontinuitet}
\subsubsection*{Definition av gränsvärden då $(x, y) \to (a, b)}
Vi säger att $f(x, y)$ har gränsvärdet $A$ då $(x, y) \to (a, b)$, och skriver
\[
    \lim_{(x, y) \to (a, b)}f(x, y) = A,
\]

\subsubsection*{Beräkning av gränsvärden då $(x, y) \to (a, b)}
I envariabelfallet finns det endast två sätt att närma sig en punkt. Med två variabler finns det oändligt många.Ett sätt att hantera är att uttrycka punkterna i polära koordinater.
\[
\begin{cases}
x = r \sin \varphi, \\
y = r \sin \varphi 
\end{cases}
\]

Tillvägagångssätt:
\begin{itemize}
    \item Gör en kvalificerad gissning av vad gränsvärdet bör vara
    \item Bilda absolutbeloppet $|f(x, y) - A|$, byt till polära koordinater, och försök att göra $|f(x,y) - A|$ oberoende av $\varphi$ genom en lämplig uppskattning uppåt.
    \item Visa att denna uppskattning går mot 0 då $r \to 0$.
\end{itemize}

\subsubsection*{Beräkning av gränsvärden då $|(x, y)| \to \infty}
(Fyll i vid behov)

\subsubsection*{Gränsvärden för allmänna funktioner $\mathbb{R}^n \to \mathbb{R}^p$}
(Fyll i vid behov)

\subsubsection*{Kontinuitet}
Låt funktionen $f$ vara en funktion av typen $\mathbb{R}^2 \to \mathbb{R}^$ som är definerad i punkten $(a, b)$. Om det gäller att
\[
    \lim_{(x,y) \to (a, b)}f(x, y) = f(a, b)
\] 
är $f$ kontinuerlig i $(a, b)$.
Antag att den reellvärda funktionen $f(x, y)$ är kontinuerlig på den slutna begränsade (dvs. kompakta) mängden D i plaet. Då antar funktionen både ett största och minsta värde i D.

\textbf{Satsen om mellanliggande värden}: Antag att den reellvärda funktionen $f(x, y)$ är kontinuerlig på en bågvis sammanhängande mängd $D$ i planet. Om $(a_1, b_1)$ och $(a_2, b_2)$ är punkter i $D$ sådana att $f(a_1, b_1) \neq f(a_2, b_2)$, så antar funktionen samtliga värden mellan $f(a_1, b_1)$ och $f(a_2, b_2)$.

\section{Differentialkalkyl}
\subsection{Partiella derivator}
\subsubsection*{Definition}
Antag att funktionen $f(x, y)$ är definierad i en omgivning av punkten $(a, b)$. 
Om gränsvärdet
\[
    \lim_{h \to 0}\frac{f(a, b+k) - f(a, b)}{h}
\]
existerar (ändligt), så säger vi att $f$ är partiellt deriverbar med avseende på $x$ i $(a, b)$. Själva gränsvärdet kallas den partiella derivatan av $f$ med avseende på $x$ i punkten $(a, b)$, och betecknas $f'_x(a, b)$

\subsubsection*{Geometrisk tolkning}
Att sätta $y$ konstant lika med $b$ innebär att geometriskt att vi skär funktionsytan $z=f(x, y)$ med planet $y=b$.
Skärningen blir en kurva. Kurvan kan ses som funktionen $z=g(x)$.

\subsubsection*{Beräkning}
\[
    f'_{x_j} (a_1, ..., a_n) = \lim_{h \to 0} \frac{f(a_n, ..., a_j + h, ... a_n) - f(a_1, ..., a_n)}{h}
\]

\subsubsection*{Tangentplan}
Ett tangentplan ges av:
\[
    z - f(a, b) = f'_x(a, b)(x-a) + f'_y(a, b)(y-b)
\]
\begin{figure}[H]
  \centering
  \includegraphics[width=0.8\textwidth]{tangentplan1.png}
  \caption{Illustration av ett tangentplan}
\end{figure}

\begin{figure}[H]
  \centering
  \includegraphics[width=0.8\textwidth]{tangentplan2.png}
  \caption{Illustration av ett tangentplans ekvation}
\end{figure}

\subsubsection*{Gradient}
Antag att funktionen $f(x, y)$ är partiellt deriverbar i punkten $(a, b)$. Vi definierar gradienten av $f$ i $(a, b)$ som vektorn
\[
    \grad f(a, b) = (f'_x(a, b), f'_y(a, b)).
\]

\subsubsection*{Riktningsderivata}
\begin{figure}[H]
  \centering
  \includegraphics[width=0.8\textwidth]{riktningsderivata.png}
  \caption{Riktningsderivata?}
\end{figure}
\[
    l_\textbf{v}(x, y) = (a, b) + t(v_1, v_2) = (a + tv_1, b + tv_2)
\]
Antag att funktionen $f$ är definierad i en omgivning av punkten $(a, b)$ och att $\textbf{v} = (v_1, v_2)$ är en vektor med längd 1. 
Vi definierar riktningsderivatan av $f$ i punkten $(a, b)$ i rikningen $\textbf{v}$, enligt
\[
    f'_v(a, b) = \lim_{t \to 0} \frac{f(a + tv_1, b+tv_2) - f(a, b)}{t}
\]
under förutsättning att detta gränsvärde existerar (ändligt).

\subsection{Differentierbarhet}
\textbf{Defintion:} Antag att funktionen $f$ är definineriad i en omgivning av punkten $(a, b)$. Vi säger att f är differentierbar i punkten $(a, b)$
om det finns tal $A$ och $B$ sådana att:
\[
    f(a + h, b + k) - f(a, b) = Ah + Bh + \sqrt{h^2 + k^2} p(h, k),  
\]
för någon funktion $p$ sådan att $p(h, k) \to 0$ då $(h, k) \to (0, 0)$.

\textbf{Sats 1:} Antag att funktionen $f(x, y)$ är differentierbar i punkten $(a, b)$. Då är $f$ partiellt deriverbar i $(a, b)$, och det gäller att:
\[
    f'_x(a, b) = A 
\]
och
\[
    f'_y(a, b) = B
\]
där $A$ och $B$ är talen i definitionen.

\textbf{Sats 2:} Om funktionen $f$ är differentierbar i punkten $(a, b)$ så är $f$ också kontinuerlig i $(a, b)$.t

\subsection{Kedjeregeln (fyll ut?)}
Antag att $g_1$ och $g_2$ är deriverbara i punkten $x$, och att $f(u, v)$ är differentierbar i punkten $(g_1(x), g_2(x))$.
Då är den sammansatta funktionen $h(x) = f(g(x)) = f(g_1(x), g_2(x))$ deriverbar i $x$ med derivatan
\[
    h'(x) = f'_u(g(x)) * g'_1(x) + f'_v(g(x))*g'_2(x)
\]

\subsection{Mer om gradient och riktningsderivata}
\textbf{Sats:} Antag att funktionen $f$ är differentierbar på det öppna området $D$, och att $D$ är bågvis sammanhängande. 
Om det gäller att grad $f(x, y) = 0$ för alla $(x, y) \in D$ så är $f$ konstant på $D$.

\textbf{Sats:} Antag att funktionen $f$ är differentierbar i punkten $(a, b)$ och att $\textbf{v} = (v_1, v_2)$ är en vektor med längd 1.
Då ges riktningsderivatan av $f$ i punkten $(a, b)$, i riktningen \textbf{v}, av skalärprodukten

\[
    f'_\textbf{v}(a, b) = \text{grad} f(a, b)*\textbf{v}
\]

\textbf{Sats:} Antag att funktionen $f$ är differentierbar i punkten $(a, b)$. Tillväxthastigheten för $f$ i $(a, b)$ varierar mellan $-|\text{grad} f(a, b)|$ och $|\text{grad} f(a, b)|$. Alltså:

\[
    -|\text{grad} f(a, b)| \leq f'_v(a, b) \leq |\text{grad} f(a, b)|
\]

\textbf{Sats:} Antag att funktionen f är kontinuerligt partiellt deriverbar i punkten $(a, b)$ och att $\text{grad} f(a, b) \neq 0$. 
Om $\Upsilon$ är nivåkurvan till $f$ genom punkten $(a, b)$ så gäller det att $\text{grad} f(a, b)$ är ortogonal mot $\Upsilon$ i $(a, b)$.

\textbf{Sats:} Antag att funktionen $f$ är kontinuerligt partiellt deriverbar i punkten $(a, b, c)$, och att $\text{grad} f(a, b, c) \neq 0$. 
Om $\Gamma$ är nivåyan till $f$ genom punkten $(a, b, c)$ så gäller det att $\text{grad} f(a, b, c)$ är ortogonal mot \Gamma i $(a, b, c)$.

\subsection{Differentialer och feluppskattning (fyll ut)}
Fyll ut.

\subsection{Högre derivator}
\[
    f''_{xy} = f''_{yx}
\]

\subsection{Kort om partiella differentialekvationer (fyll ut?)}
En differentialekvation i envariabelanalys är en ekvation som innehåller en funktioner, dess derivator, samt funktionsvariabeln.
Om vi har en ekvation med en funktion av flera variabler och dess partiella derivator, är det en partiell differentialekvation.

\section{Lokala undersökiningar och optimering}
\subsection{Taylorutveckling}
\textbf{Taylors formel:} Antag att funktionen $f(x, y)$ har kontinuerliga partiella derivator till och med ordning 3 i en omgivning av punkten $a$.
Då gäller det, för alla punker $(x, y) = (a + h, b + k)$ i denna omgivning att:\\\\
\begin{multiline}
    $f(a + h, b + k) = f(a, b) + f'_x(a, b)h + f'_y(a, b)k + \\
    \frac{1}{2}(f''_{xx}(a, b)h^2, 2f''_{xy}(a, b)hk, f''_{yy}(a, b)k^2) + (h^2 + k^2)^{\frac{3}{2}}B(h, k)$
\end{multiline}
där $B(h,k)$ är begränsad då $(h, k)$ är litet.

\subsection{Lokala extrempunkter}
\textbf{Defintion}. En punkt $(a, b) \in D_f$ kallas en \textbf{lokal maximipunkt} till funktionen $f$, 
och vi säger att $f$ har ett \textbf{lokalt maximum} i $(a, b)$ om
\[
    f(a, b) \leq f(x, y) \text{ för alla } (x, y) \in D_f \text{ nära } (a, b).
\]

\textbf{Sats}. Antag att $(a, b)$ är en lokal extrempunkt till funktionen $f$, och att $f$ är partiellt deriverbar i $(a, b)$. Då är $\text{grad} f(a, b) = 0$, dvs. alla partiella derivator är lika med noll i $(a, b)$.

\subsubsection*{Tillräckliga villkor}
Ett tillräckligt villkor garanterar att den stationära punkt vi studerar är en lokal extrempunkt.

Vi kommer också se en metod för att avgöra punktens karaktär, dvs. avgöra om den stationära punkten är maximipunkt, minimipunkt, eller ingetdeta

\textbf{Kvadratiska formen} $f$ i punkten $(a, b)$ betecknas $Q(h, k)$.
\[
    Q(h, k) = f''_{xx}(a, b)h^2 + 2f''_{xy}(a, b)hk + f''_{yy}(a, b)k^2
\]

\textbf{Definition}.
Vi säger att den kvadratiska formen $Q(h, k)$ är:
\begin{itemize}
    \item[] \textbf{positivt definit} om $Q(h, k) > 0$ för alla $(h, k) \neq (0, 0)$.
    \item[] \textbf{negativt definit} om $Q(h, k) < 0$ för alla $(h, k) \neq (0, 0)$.
    \item[] \textbf{indefinit} om $Q(h, k)$ antar både positiva och negativa värden.
\end{itemize}
Om $Q(h, k) \geq 0$ men lika med noll för något $(h, k) \neq (0, 0)$ så säger vi att formen är \textbf{positivt semidefinit},
och om $Q(h, k) \leq 0$ men lika med noll för något $(h, k) \neq (0, 0)$ så säger vi att formen är \textbf{negativt semidefinit},

\textbf{Sats}. Antag att $(a, b)$ är en stationär punkt till $f$, och att $Q(h, k)$ är den kvadratiska formen av $f$ i $(a, b)$. Då gäller:
\begin{itemize}
    \item[] om $Q(h, k)$ är positivt definit så har $f$ ett lokal minimum i $(a, b)$.
    \item[] om $Q(h, k)$ är negativt definit så har $f$ ett lokal maximum i $(a, b)$.
    \item[] om $Q(h, k)$ är indefinit så har $f$ en sadelpunkt i $(a, b)$.
\end{itemize}

I fallet då $Q(h, k)$ är semidefinit så kan vi inte dra någon slutsats om punktens karaktär.

\subsection{Optimering}

\subsubsection*{Kompakt område}
Fyll i

\subsubsection*{Icke-kompakt område}
Fyll i

\subsection{Optimering med bivillkor}
\textbf{Sats}. Antag att punkten $(a, b)$ är en (lokal) extrempunkt till funktionen $f(x, y)$ under bivillkoret $g(x, y) = C$. 
Antag vidare att $(a, b)$ är en inre punkt till $D_f$ och $D_g$. Då gäller det att $\text{grad} f(a, b)$ och $\text{grad} g(a, b)$ är parallella.

\textbf{Sats}. Antag att punkten $(a, b, c)$ är en (lokal) extrempunkt till funktionen $f(x, y, z)$ under bivillkoret $g(x, y, z) = C$. 
Antag vidare att $(a, b, c)$ är en inre punkt till $D_f$ och $D_g$. Då gäller det att $\text{grad} f(a, b, c)$ och $\text{grad} g(a, b, c)$ är parallella.

\textbf{Sats}. Antag att punkten $(a_1, ..., a_n)$ är en (lokal) extrempunkt till funktionen $f(x_1, ..., x_n)$ under bivillkoren:
\[
    g_1(x_1, ..., x_n) = C_1, g_2(x_1, ..., x_n) = C_2, g_m(x_1, ..., x_n) = C_m, 
\]

Antag vidare att $(a_1,..., a_n$ är en inre punkt till $D_f$ och $D_gi$. Då gäller det att:
\[
    \text{grad } f_1(a_1, ..., a_n), \text{grad } g_1(a_1, ..., a_n), \text{grad } g_m(a_1, ..., a_n)
\]
är linjärt beroende.

\section{Differentialkalkyl för vektorvärda funktioner}
Övn. endast för Funktionalmatris, funktionaldeterminant, Implicita funktioner.

\subsection{(Vektorvärda funktioner av en variabel)}
\subsection{(Vektorvärda funktioner av flera variabler)}

\subsection{Funktionalmatris och funktionaldeterminant}
\subsubsection*{Definition av funktionalmatris}
Låt $\textbf{f} = (f_1, f_2, ..., f_p)$ vara en funktion av variablerna $x_1, x_2, ..., x_n$.
Vi definierar funktionalmatrisen \textbf{f'} av \textbf{f} enligt: 
\[
\mathbf{f}' = \frac{\partial (f_1, f_2, \ldots, f_p)}{\partial (x_1, x_2, \ldots, x_n)} =
\begin{pmatrix}
    \frac{\partial f_1}{\partial x_1} & \frac{\partial f_1}{\partial x_2} & \cdots & \frac{\partial f_1}{\partial x_n} \\
    \frac{\partial f_2}{\partial x_1} & \frac{\partial f_2}{\partial x_2} & \cdots & \frac{\partial f_2}{\partial x_n} \\
    \vdots & \vdots & \ddots & \vdots \\
    \frac{\partial f_p}{\partial x_1} & \frac{\partial f_p}{\partial x_2} & \cdots & \frac{\partial f_p}{\partial x_n}
\end{pmatrix}
\]

\subsubsection*{Differential och linjärisering}
\[
    f(a + h) - f(a) = \Delta f = f'(a)*h + R
\]
Där restermen $R = p(h)h$ är relativt liten för små tillskott $h$.

Detta är samma för vektorvärda funktioner:

\[
\begin{pmatrix}
    \Delta f_1 \\
    \Delta f_2
\end{pmatrix}
=
\begin{pmatrix}
    \frac{\partial f_1}{\partial x} & \frac{\partial f_1}{\partial y} \\ 
    \frac{\partial f_2}{\partial x} & \frac{\partial f_2}{\partial y} \\ 
\end{pmatrix}
\begin{pmatrix}
    h \\ k
\end{pmatrix}
+
\begin{pmatrix}
    R_1 \\ R_2
\end{pmatrix}
\]

\[
    \Delta \textbf{f}' * \textbf{h} + \textbf{R}
\]

\subsubsection*{Funktionaldeterminant}
Låt $\textbf{f} = (f_1, f_2, ..., f_n)$ vara en funktion av variablerna $x_1, x_2, ..., x_n$, med funktionalmatris $\textbf{f}'$.
Vi definierar funktionaldeterminanten av $\textbf{f}'$ som

\[
\det \mathbf{f}' = 
\left|
\begin{matrix}
\frac{\partial f_1}{\partial x_1} & \frac{\partial f_1}{\partial x_2} & \cdots & \frac{\partial f_1}{\partial x_n} \\
\frac{\partial f_2}{\partial x_1} & \frac{\partial f_2}{\partial x_2} & \cdots & \frac{\partial f_2}{\partial x_n} \\
\vdots & \ddots & \vdots \\
\frac{\partial f_n}{\partial x_1} & \frac{\partial f_n}{\partial x_2} & \cdots & \frac{\partial f_n}{\partial x_n}
\end{matrix}
\right|
\]



\subsection{Inversa och implicita funktionssatsen}

\end{document}

